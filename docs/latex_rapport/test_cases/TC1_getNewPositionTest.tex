\documentclass[class=article, crop=false]{standalone}
\usepackage[subpreambles=true]{standalone}
\usepackage{import}
\usepackage[T1]{fontenc}
\usepackage[utf8]{inputenc}
\usepackage[english, danish]{babel}
\usepackage{tgbonum}
\usepackage{tabularx}
\usepackage{booktabs}
\usepackage{enumitem}
\newlist{tabenum}{enumerate}{1}
\setlist[tabenum]{label=\arabic*. ,wide=0pt, leftmargin=*, nosep, itemsep=2pt, font = \bfseries, after=\vspace{-\baselineskip}, before=\compress}
\usepackage{array, booktabs}
\makeatletter
\newcommand*{\compress}{\@minipagetrue}
\makeatother
\usepackage{paralist}
\makeatletter
\let\savespace\@minipagetrue
\makeatother



% Document
\begin{document}

    \begin{table}[H]
        \caption{Test case: TC1}
        \begin{tabularx}{\textwidth}{|l|X|}
            \hline
            \textbf{Test Case ID }       & \textbf{[TC1]: getNewPositionTest()}   \\ \hline
            \textbf{Opsummering}         & Formålet med testen er at kontrollere om getNewPosition
            metoden starter tælling forfra hver gang,
            der er nået den sidste position.\\ \hline
            \textbf{Relateret krav }     &      \\ \hline
            \textbf{Forudsætning}& Test GameBoard er oprettet med 12 felter og startpositionen er sat til 0\\ \hline
            \textbf{Udførsel}            & \begin{tabenum}
                                               \item Kalde getNewPosition metode med en defineret værdien
            \end{tabenum} \\ \hline
            \textbf{Test data}           & \begin{tabenum}
                                               \item Første tilfælde: 11
                                               \item Anden tilfælde: 12
            \end{tabenum}  \\ \hline
            \textbf{Forventet} & \begin{tabenum}
                                     \item I første tilfælde skal metoden returnere det sidste position
                                     \item Anden tilfælde skal den returenere position 0 (start position)
            \end{tabenum}  \\ \hline
            \textbf{Aktuel} & \begin{tabenum}
                                  \item Det sidste position
                                  \item Position 0 (start position)
            \end{tabenum}   \\ \hline
            \textbf{Status} & \textbf{Pass} \\ \hline
            \textbf{Lavet af} & Armandas Rokas \\ \hline
            \textbf{Dato} & 02/12/2018 \\ \hline
            \textbf{Udført af } & Armandas Rokas \\ \hline
            \textbf{Udførselses dato}  & 02/12/2018 \\ \hline
            \textbf{Test miljø}  &  \savespace \begin{compactitem}
                                                   \item IntelliJ IDEA 2018.2.2 (Ultimate Edition)
                                                   \item Build IU-182.4129.33, built on August 21, 2018
                                                   \item JRE: 1.8.0-152-release-1248-b8 amd64
                                                   \item JVM: OpenJDK 64-Bit Server VM by JetBrains s.r.o
                                                   \item Windows 10 10.0
            \end{compactitem} \\ \hline
        \end{tabularx}
    \end{table}

\end{document}