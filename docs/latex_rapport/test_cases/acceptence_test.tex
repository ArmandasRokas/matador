\documentclass[class=article, crop=false]{standalone}
\usepackage[subpreambles=true]{standalone}
\usepackage{import}
\usepackage[T1]{fontenc}
\usepackage[utf8]{inputenc}
\usepackage[english, danish]{babel}
\usepackage{tgbonum}
\usepackage{tabularx}
\usepackage{booktabs}
\usepackage{enumitem}
\newlist{tabenum}{enumerate}{1}
\setlist[tabenum]{label=\arabic*. ,wide=0pt, leftmargin=*, nosep, itemsep=2pt, font = \bfseries, after=\vspace{-\baselineskip}, before=\compress}
\usepackage{array, booktabs}
\makeatletter
\newcommand*{\compress}{\@minipagetrue}
\makeatother
\usepackage{paralist}
\makeatletter
\let\savespace\@minipagetrue
\makeatother



% Document
\begin{document}

    Følgende er vist en liste over de features, der til de forskellige use cases er godkendt som implementerede. Dette har vi testet ved alpha black box testing, hvor vi har anvendt GUI’en til at se resultatet af de forskellige testscenarier. \par
    ‘UC5: Pantsæt grund’, og ‘UC6: Ophæv pantsætning’ er medtaget i testene, da de ikke blev implementeret.

    \begin{table}[H]
        \caption{Acceptance test på UC1: Opret spillere }
        \begin{tabularx}{\textwidth}{|X|l|}
            \hline
            \textbf{Test scenarie}       & \textbf{Bestået}   \\ \hline
            Der bliver vist en læselig instruktion til oprettelse af spillere på spillepladen     & JA \\ \hline
            Der må kun accepteres input af 3 til 6 spillere    & JA \\ \hline
            Spillerne kan på tur indtaste deres navn, som bliver vist og brugt på spillepladen igennem hele spillets forløb     & JA  \\ \hline
            Spillere kan godt hede det samme, og vil bare blive tildelt forskellige farver     & JA \\ \hline
        \end{tabularx}
    \end{table}


    \begin{table}[H]
        \caption{Acceptance test på UC2: Køb grund}
        \begin{tabularx}{\textwidth}{|X|l|}
            \hline
            \textbf{Test scenarie}       & \textbf{Bestået}   \\ \hline
            Hvis spilleren er landet på en ikke-ejet ejendom, og har penge nok til at købe den, kan spilleren vælge at købe den    & JA \\ \hline
            Grunde der ejes af en spiller får en rammefarve der svarer til pågældende spillers farve på spillepladen    & JA \\ \hline
            Ejes alle ejendomme i en gruppe, fordobles lejeprisen på alle ejendommene     & JA  \\ \hline
            Der tilføjes et skøde med information om den købte ejendom til køberen     & JA \\ \hline
        \end{tabularx}
    \end{table}


    \begin{table}[H]
        \caption{Acceptance test på UC3: Kom ud af fængsel}
        \begin{tabularx}{\textwidth}{|X|l|}
            \hline
            \textbf{Test scenarie}       & \textbf{Bestået}   \\ \hline
            Spiller kan betale 50kr, for at komme ud af fængslet uden at miste den daværende tur & JA \\ \hline
            Spiller kan komme ud af fængslet ved at bruge et terningeslag til at slå 2 ens& JA \\ \hline
            Spiller kan maksimalt prøve at slå 2 ens 3 ture, hvorefter spilleren er tvunget til at betale de 50kr, for at blive løsladt. & JA \\ \hline
            Ved løsladelse ved førnævnte tvungne betaling, rykker spilleren det antal øjne, som der blev slået ved sidste forsøg på at slå 2 ens  & JA \\ \hline
        \end{tabularx}
    \end{table}

    \begin{table}[H]
        \caption{Acceptance test på UC4: Prøv lykken}
        \begin{tabularx}{\textwidth}{|X|l|}
            \hline
            \textbf{Test scenarie}       & \textbf{Bestået}   \\ \hline
            Når en spiller lander på et “Prøv lykken” felt på spillepladen, vises et tilfældigt kort fra kortsamlingen i programmet & JA \\ \hline
            Der vises en læselig besked med situationen fra lykkekorten på spillepladen & JA \\ \hline
            Spillet udfører det pågældende scenarie fra lykkekortet  & JA  \\ \hline
            Hvis spiller trækker et lykkekort, der kan løslade pågældende spiller fra fængsel, skal spilleren kunne bruge dette kort til at komme ud en gang, hvis spilleren senere er i fængsel  & JA \\ \hline
        \end{tabularx}
    \end{table}



    \begin{table}[H]
        \caption{Acceptance test på UC7: Sælg bygning }
        \begin{tabularx}{\textwidth}{|X|l|}
            \hline
            \textbf{Test scenarie}       & \textbf{Bestået}   \\ \hline
            Indsætter et korrekt beløb på spillerens konto ved salg af hus     & JA \\ \hline
            Må ikke kunne sælges, så der er ujævn fordeling     & JA \\ \hline
            Huset fjernes fra grunden på spillepladen     & JA \\ \hline
            Lejeprisen opdateres     & JA\\ \hline
            Ved salg af hotel, fjernes hotellet, og 4 huse tilføjes, på grunden på spillepladen    & JA \\ \hline
        \end{tabularx}
    \end{table}

    \begin{table}[H]
        \caption{Acceptance test på UC8: byg hus}
        \begin{tabularx}{\textwidth}{|X|l|}
            \hline
            \textbf{Test scenarie}       & \textbf{Bestået}   \\ \hline
            Må kun bygges på grunde, hvor alle ejendomme i gruppen er ejer af den samme spiller     & JA \\ \hline
            Tilføjer en hus på grunden på spillepladen     & JA\\ \hline
            Maksimum 4 huse må bygges på en grund     & JA\\ \hline
            Må ikke kunne bygges ujævnt     & JA\\ \hline
            Trækkes et korrekt beløb fra spillerens konto ved køb af hus     & JA\\ \hline
            Lejeprisen opdateres     & JA \\ \hline
        \end{tabularx}
    \end{table}


    \begin{table}[H]
        \caption{Acceptance test på UC9: byg hotel}
        \begin{tabularx}{\textwidth}{|X|l|}
            \hline
            \textbf{Test scenarie}       & \textbf{Bestået}   \\ \hline
            Må kun bygges på en grund ejet af en spiller med nok penge og 4 huse på alle grunde i den pågældende grunds gruppe     & JA \\ \hline
            Fjerner de 4 huse og tilføjer et hotel på grunden     & JA\\ \hline
            Der må maksimum være et hotel på hver grund     & JA\\ \hline
            Der trækkes et korrekt beløb fra spillerens konto ved køb af hotel     & JA\\ \hline
            Lejeprisen opdateres     & JA \\ \hline
        \end{tabularx}
    \end{table}

    \begin{table}[H]
        \caption{Acceptance test på UC10: gå fallit}
        \begin{tabularx}{\textwidth}{|X|l|}
            \hline
            \textbf{Test scenarie}       & \textbf{Bestået}   \\ \hline
            Overføres grundene, som er ejet af spilleren til en kreditor (enten den anden spiller, som spilleren, der går fallit skylder penge, eller banken) & JA\\ \hline
            Grundenes rammefarve opdateres til den nye ejers farve på spillepladen & JA\\ \hline
            Lejeprisen på felter updates, hvis de overførte grunde fra en spiller, der er gået fallit gør at alle felter i en gruppe ejes (opdateres også på spillepladen) & JA\\ \hline
            Hvis alle undtagen en spiller er gået fallit, udnævnes en vinder& JA\\ \hline
            En spiller må ikke længere interegere længere med spillet, efter spilleren er gået fallit. Inklusive at administrere ejendomme.& JA \\ \hline
        \end{tabularx}
    \end{table}


\end{document}