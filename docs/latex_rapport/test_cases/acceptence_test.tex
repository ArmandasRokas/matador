\documentclass[class=article, crop=false]{standalone}
\usepackage[subpreambles=true]{standalone}
\usepackage{import}
\usepackage[T1]{fontenc}
\usepackage[utf8]{inputenc}
\usepackage[english, danish]{babel}
\usepackage{tgbonum}
\usepackage{tabularx}
\usepackage{booktabs}
\usepackage{enumitem}
\newlist{tabenum}{enumerate}{1}
\setlist[tabenum]{label=\arabic*. ,wide=0pt, leftmargin=*, nosep, itemsep=2pt, font = \bfseries, after=\vspace{-\baselineskip}, before=\compress}
\usepackage{array, booktabs}
\makeatletter
\newcommand*{\compress}{\@minipagetrue}
\makeatother
\usepackage{paralist}
\makeatletter
\let\savespace\@minipagetrue
\makeatother



% Document
\begin{document}

    \begin{table}[H]
        \caption{Acceptance test på UC1: Opret spillere }
        \begin{tabularx}{\textwidth}{|X|l|}
            \hline
            \textbf{Test scenarie}       & \textbf{Bestået}   \\ \hline
            Der bliver vist en læselig instruktion til oprettelse af spillere på spillepladen     & - \\ \hline
            Der må kun accepteres input af 3 til 6 spillere    & - \\ \hline
            Spillerne kan på tur indtaste deres navn, som bliver vist og brugt på spillepladen igennem hele spillets forløb     & -  \\ \hline
            Spillere kan godt hede det samme, og vil bare blive tildelt forskellige farver     & - \\ \hline
        \end{tabularx}
    \end{table}


    \begin{table}[H]
        \caption{Acceptance test på UC2: Køb grund}
        \begin{tabularx}{\textwidth}{|X|l|}
            \hline
            \textbf{Test scenarie}       & \textbf{Bestået}   \\ \hline
            Hvis spilleren er landet på en ikke-ejet ejendom, og har penge nok til at købe den, kan spilleren vælge at købe den    & - \\ \hline
            Grunde der ejes af en spiller får en rammefarve der svarer til pågældende spillers farve på spillepladen    & - \\ \hline
            Ejes alle ejendomme i en gruppe, fordobles lejeprisen på alle ejendommene     & -  \\ \hline
            Der tilføjes et skøde med information om den købte ejendom til køberen     & - \\ \hline
            Auktion    & NEJ \\ \hline
        \end{tabularx}
    \end{table}


    \begin{table}[H]
        \caption{Acceptance test på UC3: Kom ud af fængsel}
        \begin{tabularx}{\textwidth}{|X|l|}
            \hline
            \textbf{Test scenarie}       & \textbf{Bestået}   \\ \hline
            Spiller kan betale 50kr, for at komme ud af fængslet uden at miste den daværende tur & - \\ \hline
            Spiller kan komme ud af fængslet ved at bruge et terningeslag til at slå 2 ens& - \\ \hline
            Spiller kan maksimalt prøve at slå 2 ens 3 ture, hvorefter spilleren er tvunget til at betale de 50kr, for at blive løsladt. & -  \\ \hline
            Ved løsladelse ved førnævnte tvungne betaling, rykker spilleren det antal øjne, som der blev slået ved sidste forsøg på at slå 2 ens  & - \\ \hline
        \end{tabularx}
    \end{table}

    \begin{table}[H]
        \caption{Acceptance test på UC4: Prøv lykken}
        \begin{tabularx}{\textwidth}{|X|l|}
            \hline
            \textbf{Test scenarie}       & \textbf{Bestået}   \\ \hline
            Når en spiller lander på et “Prøv lykken” felt på spillepladen, vises et tilfældigt kort fra kortsamlingen i programmet & - \\ \hline
            Der vises en læselig besked med situationen fra lykkekorten på spillepladen & - \\ \hline
            Spillet udfører det pågældende scenarie fra lykkekortet  & -  \\ \hline
            Hvis spiller trækker et lykkekort, der kan løslade pågældende spiller fra fængsel, skal spilleren kunne bruge dette kort til at komme ud en gang, hvis spilleren senere er i fængsel  & - \\ \hline
        \end{tabularx}
    \end{table}



    \begin{table}[H]
        \caption{Acceptance test på UC7: Sælg bygning }
        \begin{tabularx}{\textwidth}{|X|l|}
            \hline
            \textbf{Test scenarie}       & \textbf{Bestået}   \\ \hline
            Indsætter et korrekt beløb på spillerens konto ved salg af hus     & - \\ \hline
            Må ikke kunne sælges, så der er ujævn fordeling     & - \\ \hline
            Huset fjernes fra grunden på spillepladen     & - \\ \hline
            Lejeprisen opdateres     & -\\ \hline
            Ved salg af hotel, fjernes hotellet, og 4 huse tilføjes, på grunden på spillepladen    & - \\ \hline
        \end{tabularx}
    \end{table}

    \begin{table}[H]
        \caption{Acceptance test på UC8: byg hus}
        \begin{tabularx}{\textwidth}{|X|l|}
            \hline
            \textbf{Test scenarie}       & \textbf{Bestået}   \\ \hline
            Må kun bygges på grunde, hvor alle ejendomme i gruppen er ejer af den samme spiller     & - \\ \hline
            Tilføjer en hus på grunden på spillepladen     & -\\ \hline
            Maksimum 4 huse må bygges på en grund     & -\\ \hline
            Må ikke kunne bygges ujævnt     & -\\ \hline
            Trækkes et korrekt beløb fra spillerens konto ved køb af hus     & -\\ \hline
            Lejeprisen opdateres     & - \\ \hline
        \end{tabularx}
    \end{table}


    \begin{table}[H]
        \caption{Acceptance test på UC9: byg hotel}
        \begin{tabularx}{\textwidth}{|X|l|}
            \hline
            \textbf{Test scenarie}       & \textbf{Bestået}   \\ \hline
            Må kun bygges på en grund ejet af en spiller med nok penge og 4 huse på alle grunde i den pågældende grunds gruppe     & - \\ \hline
            Fjerner de 4 huse og tilføjer et hotel på grunden     & -\\ \hline
            Der må maksimum være et hotel på hver grund     & -\\ \hline
            Der trækkes et korrekt beløb fra spillerens konto ved køb af hotel     & -\\ \hline
            Lejeprisen opdateres     & - \\ \hline
        \end{tabularx}
    \end{table}

    \begin{table}[H]
        \caption{Acceptance test på UC10: gå fallit}
        \begin{tabularx}{\textwidth}{|X|l|}
            \hline
            \textbf{Test scenarie}       & \textbf{Bestået}   \\ \hline
            Overføres grundene, som er ejet af spilleren til en kreditor (enten den anden spiller, som spilleren, der går fallit skylder penge, eller banken) & - \\ \hline
            Grundenes rammefarve opdateres til den nye ejers farve på spillepladen & -\\ \hline
            Lejeprisen på felter updates, hvis de overførte grunde fra en spiller, der er gået fallit gør at alle felter i en gruppe ejes (opdateres også på spillepladen) & -\\ \hline
            Hvis alle undtagen en spiller er gået fallit, udnævnes en vinder& -\\ \hline
        \end{tabularx}
    \end{table}


\end{document}